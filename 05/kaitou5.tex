\documentclass[dvipdfmx]{jsarticle}
\usepackage{amsmath,amssymb}
\usepackage{amsthm}
\usepackage{bm}

\usepackage{tcolorbox}
\begin{document}
\begin{tcolorbox}[colframe=black!50,colback=white,colbacktitle=black!50,coltitle=white,fonttitle=\bfseries\sffamily,title=問題1]
$n \in \mathbb{Z}に対して、\mathbb{Z}_nを考える. \ \mathbb{Z}_nが話に関して群になることはよく知られている。$
\begin{enumerate}
  \item $\mathbb{Z}_3,および\mathbb{Z}_4の積に関する表を作り、どちらも環となることを確認せよ.$
  \item $一般のn \in \mathbb{N}に対して、(\mathbb{Z}_n, +, \times)が環になることを確認せよ.$
  \item $\mathbb{Z}_3,\mathbb{Z}_4はそれぞれ整域であるかどうかを判定せよ.$
  \item $\mathbb{Z}_3, \mathbb{Z}_4はそれぞれ体であるかどうか判定せよ.$
  \item $体は整域であることを示せ.$
  \item $素数pに対して、(\mathbb{Z}_p, +, \times)は体になることを示せ.$
\end{enumerate}
\end{tcolorbox}

\begin{description}
  \item[解答] \mbox{}
  \begin{enumerate}
    \item 略

    \item 略

    \item
    \begin{itemize}
      \item $\mathbb{Z}_3について$\\
      $3は素数であり、問題6から体.したがって、整域$

      \item $\mathbb{Z}_4について$\\
      $[2] \in \mathbb{Z}_4は[2] \cdot [2] = [4] = [0]. \ よって、[2]は[2]の零因子なので、\mathbb{Z}_4は整域でない.$

    \end{itemize}

    \item
    \begin{itemize}
      \item $\mathbb{Z}_3について$

      \item $\mathbb{Z}_4について$\\
      $\mathbb{Z}_4は整域でないので、体でもない.$

    \end{itemize}

    \item $1$
  \end{enumerate}
\end{description}


\begin{tcolorbox}[colframe=black!50,colback=white,colbacktitle=black!50,coltitle=white,fonttitle=\bfseries\sffamily,title=問題2]
$Rを環、そのイデアルをJとする$
\begin{enumerate}
  \item $Jがイデアルであることの定義をかけ$
  \item $イデアルJに単位元が含まれれば、J=Rとなることを示せ$
  \item $体には自明なイデアルしかないことを示せ$
  \item $剰余環R/Jにおける加法と乗法の定義を書け.また、乗法が矛盾なく定義できることを示せ$
\end{enumerate}
\end{tcolorbox}


\begin{description}
  \item[解答] \mbox{}\\
  \begin{enumerate}
    \item Rの部分集合Jがイデアルとは以下の二つを満たしてることである
    \begin{enumerate}
      \item $JがRの加法に関する部分群$
      \item $\forall a \in J, \ \forall r \in R,\ r \cdot a \in J.$
    \end{enumerate}

    \item $J \subset Rは明らか$ \\
    $R \subset Jを示す。$\\
    $1 \in Jであることから、\forall r \in Rに対して、r = r \cdot 1 \in J.よって、R \subset J.$

    \item $Kを体とし,I \subset Kをイデアルであり、I \neq \emptyset とする$\\
    $a \in Iという元が存在してそれを固定する。1 = a^{-1} \cdot a \in I. よって問題3よりI = K $\\
    $以上からKには自明なイデアルしかない$

    \item $x+J,y+J \in R/Jに対して$
    \begin{itemize}
      \item 加法\\
      $(x+J) + (y+J) = (x+y) + J$
      つまり、$[x]+[y] = [x+y]$
      \item 積\\
      $(x+J)(y+J) = (xy)+J$
      $つまり、[x] \cdot [y] = [xy]$

      \item well-defined性について
      \begin{itemize}
        \item イデアルはアーベル群であるから正規部分群であるので、加法にたいしてはwell-defined

        \item 積について\\
        $a \in [x],b \in [y]とする。したがって、\exists j_1,j_2 \in J \ s.t. \ a = x + j_1, b = y + j_2. \\$
        $[a] \cdot [b] = [ab] = [(x+j_1)(y+j_2)] = [xy+xj_2+yj_1+j_1j_2] = [xy]. \ (Jはイデアルであることから、xj_2+yj_1+j_1j_2 \in J)$\\
        $よって、well-defined$

      \end{itemize}
    \end{itemize}
  \end{enumerate}
\end{description}

\begin{tcolorbox}[colframe=black!50,colback=white,colbacktitle=black!50,coltitle=white,fonttitle=\bfseries\sffamily,title=問題3]
$環 \mathbb{Z}の部分集合\{3059,4807\}に対して、それによって生成されるイデアルJを以下で定める$\\
$J=(3059,4807) = \{3059r_1 + 4807r_2 \mid r_1,r_2 \in \mathbb{Z} \}$\\
$Jは単項イデアルであることを示したい$
\begin{enumerate}
  \item $3059と4807の最大公約数を求めよ$
  \item $(1)で求めた最大公約数をgとし、gの生成する\mathbb{Z}上のイデアルKを以下で定めると、K \subset Jとなることを示せ$\\
  $K = (g) = \{ gr \mid r \in \mathbb{Z} \} $
  \item $J \subset Kとなることを示せ$
\end{enumerate}
\end{tcolorbox}



\begin{tcolorbox}[colframe=black!50,colback=white,colbacktitle=black!50,coltitle=white,fonttitle=\bfseries\sffamily,title=問題4]
$R,Sを環として、\phi: R \rightarrow Sを環の準同型とする$
\begin{enumerate}
  \item $\phi(0) = 0, \phi(-a) = -\phi(a)を確かめよ$
  \item $Im(\phi)はSの部分環となることを示せ$
  \item $Rが体の時、\phi は零写像でなければ単射であることを示せ$
\end{enumerate}
\end{tcolorbox}

\begin{description}
  \item[解答] \mbox{}\\
  \begin{enumerate}
    \item $\phi は群の準同型でもあるので明らか$
    \item $Im(\phi)は \phi の群準同型の性質から明らかにSの和に関する部分群$\\
    $\forall a,b \in Im(\phi) に対して、\exists x,y \in R \ s.t. \ a = \phi(x), b = \phi(y).$ \\
    $ a \cdot b = \phi(x) \cdot \phi(y) = \phi(xy) \in Im(\phi).$\\
    $1_S = \phi(1_R) \in Im(\phi). \ \ 以上から、Im(\phi)はSの部分環$

    \item $Ker(\phi) = {0}を示す.$\\
    $Ker(\phi)はRのイデアルであり、Rは体なのでRのイデアルは自明なものしかない。$\\
    $したがって、Ker(\phi) = (0_R) \ or \ R.しかし、\phi(1_R) = 1_S \neq 0_S \ よりKer(\phi) \neq R.したがって、Ker(\phi) = (0_R). $
  \end{enumerate}
\end{description}
\end{document}
