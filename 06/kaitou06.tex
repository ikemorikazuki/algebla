\documentclass[dvipdfmx]{jsarticle}
\usepackage{amsmath,amssymb}
\usepackage{amsthm}
\usepackage{bm}
\usepackage{graphicx}
\usepackage{cases}

\usepackage{tcolorbox}
\begin{document}

\begin{tcolorbox}[colframe=black!50,colback=white,colbacktitle=black!50,coltitle=white,fonttitle=\bfseries\sffamily,title=問題1]
$多項式環 \mathbb{Q}[x] は単項イデアル整域であることが知られている。\mathbb{Q}[x]上のイデアル(x^5+2x^4-4x^3-8x^2+3x+6,x^4+x^3-9x^2-3x^2+18)を生成する元を求めよ$
\end{tcolorbox}

\begin{description}
  \item[解答] \mbox{}\\
  $演習No.5より、(x^5+2x^4-4x^3-8x^2+3x+6,x^4+x^3-9x^2-3x^2+18)は最大公約元によって生成される.最大公約元はユークリッドの互除法によって求められる$\\
  $gcd(x^5+2x^4-4x^3-8x^2+3x+6,x^4+x^3-9x^2-3x^2+18) \\
  = gcd(x^4+x^3-9x^2-3x^2+18,4x^3+4x^2-12x-12) \\
  = gcd(x^4+x^3-9x^2-3x^2+18,x^3+x^2-3x-3) \\
  = gcd(x^3+x^2-3x-3,-6x^2+18) \\
  = gcd(x^3+x^2-3x-3, -6x^2+18) \\
  = gcd(x^3+x^2-3x-3, x^2-3) \\
  = gcd(1, x^2-3)$ \\
  $よって、生成元は(x^2-3)$
\end{description}


\begin{tcolorbox}[colframe=black!50,colback=white,colbacktitle=black!50,coltitle=white,fonttitle=\bfseries\sffamily,title=問題2]
$\mathbb{Q}(\sqrt{3}) = \{ a+b \sqrt{3} \mid a,b \in \mathbb{Q} \}と\mathbb{Q}[x]/(x^2-3)が環として同型であることを示したい$\\
$\phi : \mathbb{Q}[x] \rightarrow \mathbb{Q}(\sqrt{3})をf(x)に対して、f(\sqrt{3})を対応させるものとする。$

\begin{enumerate}
  \item $\phi が環準同型であることを示せ$
  \item $\phi が全射であることを示せ$
  \item $Ker\phi が(x^2-3)であることを示せ$
  \item $環準同型定理を用いて、同型であることを示せ$
\end{enumerate}
\end{tcolorbox}

\begin{description}
  \item[解答] \mbox{}
  \begin{enumerate}
    \item $\forall f, g \in \mathbb{Q}[x]に対して、\phi(f(x)+g(x)) = f(\sqrt{3}) + g(\sqrt{3}) = \phi(f(x)) + \phi(g(x))$ \\
    $\forall f,g \in \mathbb{Q}[x]に対して、\phi (f(x)g(x)) = f(\sqrt{3})g(\sqrt{3}) = \phi(f(x))\phi(g(x))$ \\
    $\phi(1) = 1$ \\
    $よって,\phi は環準同型$

    \item $\forall a+b\sqrt{3} \in \mathbb{Q}(\sqrt{3})にたして、a+bx \in \mathbb{Q}[x]によって、\phi(a+bx) = a+b\sqrt{3}となるので、\phi は全射$

    \item $Ker(\phi) = (x^2-3)を示す$
    \begin{itemize}
      \item $Ker(\phi) \subset (x^2-3)$\\
      $\forall f(x) \in Ker(\phi)に対して、\exists g(x) \in \mathbb{Q}[x], \ \exists a, b \in \mathbb{Q} \ s.t. \ f(x) = (x^2-3)g(x) + ax+b$ \\
      $f(\sqrt{3}) = a\sqrt{3}+b = 0. \ したがって、a=b=0. ゆえに、f(x) \in (x^2-3)$

      \item $(x^2-3) \subset Ker(\phi)$\\
      $\forall f[x] \in (ker(\phi))に対して、\exists g(x) \mathbb{Q}[x] \ s.t. \ f(x) = (x^2-3)g(x). \ したがって、f(\sqrt{3}) = (3-3)g(x) = 0. \ よって、f(x) \in Ker(\phi)$
    \end{itemize}
    $以上から、Ker(\phi) = (x^2-3)$

    \item $以上より、準同型定理から\mathbb{Q}(\sqrt{3}) \simeq \mathbb{Q}[x]/(x^2-3)$
  \end{enumerate}
\end{description}

\begin{tcolorbox}[colframe=black!50,colback=white,colbacktitle=black!50,coltitle=white,fonttitle=\bfseries\sffamily,title=問題3]
$\mathbb{Z}_2[t]をt^2+t+1の生成するイデアルで割った剰余環R=\mathbb{Z}_2[t]/(t^2+t+1)を考えるとこれは有限になる$
\begin{enumerate}
  \item $Rの要素を列挙せよ$
  \item $Rの和に関する表と積に関する表を書き、Rが体であることかどうかを判定せよ$
  \item $Rの位数をkとする。Rと\mathbb{Z}_kは同型であるか判定せよ$
\end{enumerate}
\end{tcolorbox}

\begin{description}
  \item[解答] \mbox{}\\
  \begin{enumerate}
    \item $\{ \overline{0}, \ \overline{1},\ \overline{t}, \ \overline{t+1} \} $

    \item $表は略、Rは体$

    \item $Rは\mathbb{Z}_4と同型ではない$\\
    $\because \ Rは体だが\mathbb{Z}_4は体ではないため$
  \end{enumerate}

\end{description}

\begin{tcolorbox}[colframe=black!50,colback=white,colbacktitle=black!50,coltitle=white,fonttitle=\bfseries\sffamily,title=問題4]
$体K上の多項式環K[x]はユークリッド整域であることを示せ.$
\end{tcolorbox}

\begin{description}
  \item[解答] \mbox{}\\
  $写像\deg を以下のように定義する$ \\
  $$
\begin{array}{cccc}
  d: & K[x] \setminus \{0\}  & \longrightarrow & \mathbb{N} \\
  & \rotatebox{90}{$\in$} &     &\rotatebox{90}{$\in$}\\
  & f(x) & \longmapsto & deg(f)
\end{array}
$$
$f(x), \ g(x) \in K[x],g(x) \neq 0とする$\\
$すると、多項式の割り算の定義から \ \exists ! \  q(x), \ r(x) \in K[x] \ s.t. \ f(x) = q(x)g(x) + r(x) \ (deg(r) < deg(g)).したがって、K[x]はユークリッド環 $

\end{description}

\begin{tcolorbox}[colframe=black!50,colback=white,colbacktitle=black!50,coltitle=white,fonttitle=\bfseries\sffamily,title=問題5]
一意分解整域について以下の問いに答えよ
\begin{enumerate}
  \item $\mathbb{Z}は一意分解整域であることを説明せよ$
  \item $\mathbb{Z}[\sqrt{-5}] = \{ a+b\sqrt{-5} \mid a,b \in \mathbb{Z} \}は一意分解整域でないことを示したい$
  \begin{enumerate}
    \item $N:\mathbb{Z}[\sqrt{-5}] \rightarrow \mathbb{Z}をN(a+b\sqrt{-5}) = a^2+5b^2で定める。\alpha,\beta \in \mathbb{Z}[\sqrt{-5}]に対して、N(\alpha) \geq 0 \ および、N(\alpha \beta) = N(\alpha)N(\beta)を示せ $
    \item $N(\alpha) \leq 6となる\alpha \in \mathbb{Z}[\sqrt{-5}]を列挙せよ$
    \item $\mathbb{Z}[\sqrt{-5}]において、6をふた通りの積に分解せよ$
    \item $全問の分解が既約元への分解になっていることを示せ$
  \end{enumerate}
\end{enumerate}
\end{tcolorbox}

\begin{description}
  \item[解答]
  \begin{enumerate}
    \item $\mathbb{Z}の素数は素元であり、\mathbb{Z}は素因数分解によって素元の積に分解され、また素因数分解の一意性から順番を無視して一意になる$

    \item $\alpha = a+b\sqrt{-5}, \beta = c+d\sqrt{-5}とおく N(\alpha) = a^2+5b^2 \geq 0, また、N(\alpha)N(\beta) = (a^2+5b^2)(c^2+5d^2) = (ac-5bd)^2+5(ad+cb)^2 = N(\alpha\beta)$

    \item $\alpha = a+\sqrt{-5}bとする。N(\alpha) = a^2+5b^2.$\\
    $a^2+5b^2 \leq 6 より-1 \leq b \leq 1. $\\
    $b = 0のとき、a^2 \leq 6 \Leftrightarrow -\sqrt{6} \leq a \leq  \sqrt{6}.したがって、a = 0 ,\pm 1, \pm 2 である。$\\
    $b = \pm 1のとき、a^2 + 5 \leq 6 \Leftrightarrow -1 \leq a \leq 1 \Leftrightarrow a = 0, \ \pm 1$\\
    $以上から、\alpha = 0,\pm 1,\  \pm 2, \ \pm 1 \pm \sqrt{-5} \ (複合任意)$

    \item $6 = 2 \cdot 3 = (1-\sqrt{-5})(1+\sqrt{-5})$

    \item $2は\mathbb{Z}[\sqrt{-5}]において既約元であることを示す$\\
    $x,y \in \mathbb{Z}[\sqrt{-5}]によって、2 = xyと表されたとする。すると、N(xy) = N(x)N(y) = N(2) = 4となる。N(x),N(y) \in \mathbb{Z}より、『N(x) = 2 かつ  N(y) = 2』または『N(x)=1 かつ N(y)=4』となる。(対称性を考慮している)。N(x) = a^2+5b^2=2となるa,b \in \mathbb{Z}は存在しないので、「N(x)=1 かつ N(y)=4」 \Leftrightarrow 「x = \pm 1 かつ y = \pm 2」.したがって、2は既約元.$\\
    $3も既約元であることも同様に示せる.$\\
    $1+\sqrt{-5}は\mathbb{Z}[\sqrt{-5}]において既約元であることを示す$\\
    $x,y \in \mathbb{Z}[\sqrt{-5}]によって、1+\sqrt{-5} = xyと表されたとする。すると、N(xy) = N(x)N(y) = N(1+\sqrt{-5}) = 6となる。N(x),N(y) \in \mathbb{Z}より、『N(x) = 2 かつ  N(y) = 3』または『N(x)=1 かつ N(y)=6』となる。(対称性を考慮している)。N(x) = a^2+5b^2=2となるa,b \in \mathbb{Z}は存在しないので、「N(x)=1 かつ N(y)=6」 \Leftrightarrow 「x = \pm 1 かつ y = \pm 11+\sqrt{-5}」.したがって、1+\sqrt{-5}は既約元.$\\
    $1-\sqrt{-5}も既約元であることも同様に示せる.$\\
    $以上から、問題4と合わせて6が規約元の積にふた通りに表されることが示された。$
  \end{enumerate}
\end{description}


\end{document}
