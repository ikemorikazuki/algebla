\documentclass[dvipdfmx]{jsarticle}
\usepackage{amsmath,amssymb}
\usepackage{amsthm}
\usepackage{bm}
\usepackage{graphicx}
\usepackage{cases}

\usepackage{tcolorbox}
\begin{document}

\begin{tcolorbox}[colframe=black!50,colback=white,colbacktitle=black!50,coltitle=white,fonttitle=\bfseries\sffamily,title=問題1]
$拡大体\mathbb{Q}(\sqrt{3}) / \mathbb{Q}について、以下の問いに答えよ$
\begin{enumerate}
  \item $\mathbb{Q}(\sqrt{3}) / \mathbb{Q} = \{a+b\sqrt{3} \mid a,b \in \mathbb{Q} \}であることを確認せよ.また拡大次数[\mathbb{Q}(\sqrt{3}) ; \mathbb{Q}]を求めよ.$
  \item $\mathbb{Q}(\sqrt{3}) / \mathbb{Q}のガロア群Gal(\mathbb{Q}(\sqrt{3}) / \mathbb{Q})を求めたい.$
  \begin{enumerate}
    \item $f を \mathbb{Q}(\sqrt{3}) / \mathbb{Q}の自己同型写像とする. \ f(1)およびf(\sqrt{3})が決まれば、写像f自体が定まることを示せ.$
    \item $fが\mathbb{Q}(\sqrt{3}) / \mathbb{Q}の自己同型写像となるための、f(1)およびf(\sqrt{3})に関する条件を求め\mathbb{Q}(\sqrt{3}) / \mathbb{Q}の自己同型写像を列挙せよ。$
    \item $\mathbb{Q}(\sqrt{3}) / \mathbb{Q}^{Gal(\mathbb{Q}(\sqrt{3}) / \mathbb{Q})}を求めよ。$
    \item $\mathbb{Q}(\sqrt{3}) / \mathbb{Q}はガロア拡大かどうか判定せよ。$
  \end{enumerate}
\end{enumerate}
\end{tcolorbox}

\begin{description}
  \item[解答] \mbox{}
  \begin{enumerate}
    \item $f(x)=x^2-3 \ \in \mathbb{Q}[x]. \ f(x)はアイゼンシュタインの判定法より \mathbb{Q}[x]上の既約多項式であり、\sqrt{3} \ \not \in \mathbb{Q}より、\sqrt{3}の\mathbb{Q}[x]上の最小多項式の次数は2以上. \ よって、f(x)は\mathbb{Q}[x]上の最小多項式.$ \\
    $したがって、\mathbb{Q}(\sqrt{3})は \mathbb{Q}上のベクトル空間としての次元は2であり、基底として\{1,\sqrt{3}\}が取れるので、\mathbb{Q}(\sqrt{3}) = \{ a+b\sqrt{3} \mid a,b \in \mathbb{Q} \}$ \\ $[\mathbb{Q}(\sqrt{3}) ; \mathbb{Q}] = 2$

    \item
    \begin{enumerate}
      \item $\forall a+b\sqrt{3}に対して、fが\mathbb{Q}-自己同型写像であることに注意して、 f(a+b\sqrt{3}) = f(a) + f(\sqrt{3}) = a \cdot f(1) + b \cdot f(\sqrt{3}). \ よって、f(1),f(\sqrt{3})によってfが定まる。$

      \item $fが\mathbb{Q}-自己同型写像であることから,f(1) = 1.$\\
      $f(\sqrt{3}) \cdot f(\sqrt{3}) = f(\sqrt{3}^2) = f(3) = 3. \ よって、f(\sqrt{3}) = \pm \sqrt{3}.$\\
      $以上から、『f(1)=1』かつ『f(\sqrt{3}) = \pm \sqrt{3}』$\\
      $よって、Gal(\mathbb{Q}(\sqrt{3}) / \mathbb{Q}) = \{id,\sigma \}. \ (ただし,\sigma (\sqrt{3}) = -\sqrt{3} となる写像)$

      \item $\mathbb{Q}(\sqrt{3}) / \mathbb{Q}^{Gal(\mathbb{Q}(\sqrt{3}) / \mathbb{Q})} = \{ \alpha \in \mathbb{Q}(\sqrt{3}) \ \mid \forall \sigma \in Gal(\mathbb{Q}(\sqrt{3}) / \mathbb{Q}), \  \sigma(\alpha) = \alpha \} = \mathbb{Q}を示す。$
      \begin{itemize}
        \item $\subset$\\
        $\forall \alpha = a+b\sqrt{3} \in \mathbb{Q}(\sqrt{3}) / \mathbb{Q}^{Gal(\mathbb{Q}(\sqrt{3}) / \mathbb{Q})}に対して、\sigma(a+b\sqrt{3}) = a-b\sqrt{3}より、$\\
        $a+b\sqrt{3} = a-b\sqrt{3}. \ \ \ (\because \mathbb{Q}(\sqrt{3}) / \mathbb{Q}^{Gal(\mathbb{Q}(\sqrt{3}) / \mathbb{Q})}の定義) $\\
        $\Leftrightarrow 2b\sqrt{3} = 0$\\
        $\Leftrightarrow b=0$\\
        $よって、\alpha = a \in \mathbb{Q}$
        \item $\supset は明らか$
      \end{itemize}
      $以上から、\mathbb{Q}(\sqrt{3}) / \mathbb{Q}^{Gal(\mathbb{Q}(\sqrt{3}) / \mathbb{Q})} = \mathbb{Q}$

      \item $\mathbb{Q}(\sqrt{3}) / \mathbb{Q}^{Gal(\mathbb{Q}(\sqrt{3}) / \mathbb{Q})} = \mathbb{Q}より、ガロア拡大$\\
      $\mid \mathbb{Q}(\sqrt{3}) / \mathbb{Q}^{Gal(\mathbb{Q}(\sqrt{3}) / \mathbb{Q})} \mid = [\mathbb{Q}(\sqrt{3}) ; \mathbb{Q}] = 2からガロア拡大と言ってもよい$
    \end{enumerate}
  \end{enumerate}
\end{description}

\begin{tcolorbox}[colframe=black!50,colback=white,colbacktitle=black!50,coltitle=white,fonttitle=\bfseries\sffamily,title=問題2]
$ガロア理論の基本定理を\mathbb{Q}(\sqrt{2},\sqrt{3}) / \mathbb{Q}について、確認したい$
\begin{enumerate}
  \item $\mathbb{Q}(\sqrt{2},\sqrt{3}) / \mathbb{Q}の拡大次数を求めよ。$
  \item $Gal(\mathbb{Q}(\sqrt{2},\sqrt{3}) / \mathbb{Q})を求めよ。$
  \item $中間体\mathbb{Q}(\sqrt{2}) / \mathbb{Q}に対応する。Gal(\mathbb{Q}(\sqrt{2},\sqrt{3}) / \mathbb{Q})の部分群を求めよ。$
  \item $f \in Gal(\mathbb{Q}(\sqrt{2},\sqrt{3}) / \mathbb{Q})をf(1)=1, \ f(\sqrt{2}) = -\sqrt{2}, \ f(\sqrt{3}) = \sqrt{3}で定める自己同型写像とする。\{id,f\}はGal(\mathbb{Q}(\sqrt{2},\sqrt{3}) / \mathbb{Q})の部分群となるが、これに対応する中間体を求めよ。$
\end{enumerate}
\end{tcolorbox}

\begin{description}
  \item[解答] \mbox{}
  \begin{enumerate}
    \item
    \begin{itemize}
      \item $f(x) = x^2-2は \ \sqrt{2} \not \in \mathbb{Q}より、\mathbb{Q}上の最小多項式. \ よって、[\mathbb{Q}(\sqrt{2}) ; \mathbb{Q}] = 2$
      \item $g(x) = x^2-3は \ \sqrt{3} \not \in \mathbb{Q}(\sqrt{2})より、\mathbb{Q}(\sqrt{2})上の最小多項式。 \ よって、[\mathbb{Q}(\sqrt{2}, \ \sqrt{3}) ; \mathbb{Q}(\sqrt{2})] = 2$
    \end{itemize}
    $以上から、[\mathbb{Q}(\sqrt{2},\sqrt{3}); \mathbb{Q}] = [\mathbb{Q}(\sqrt{2},\sqrt{3}) ; \mathbb{Q}(\sqrt{2})] \  [\mathbb{Q}(\sqrt{2}) ; \mathbb{Q}] = 4.$

    \item $Gal(\mathbb{Q}(\sqrt{2},\sqrt{3}) / \mathbb{Q})について$
    \begin{itemize}
      \item $まず、\mathbb{Q}(\sqrt{2},\sqrt{3}) / \mathbb{Q}がガロア拡大であることを示す$\\
      $\sqrt{2} \not \in \mathbb{Q}より、\sqrt{2}の\mathbb{Q}上の最小多項式はx^2-2であり、\mathbb{Q}上の共役は \pm \sqrt{2}.$\\
      $\sqrt{3} \not \in \mathbb{Q}より、\sqrt{3}の\mathbb{Q}上の最小多項式はx^2-3であり、\mathbb{Q}上の共役は \pm \sqrt{3}$\\
      $\pm \sqrt{3},\pm \sqrt{2} \in \mathbb{Q}(\sqrt{2},\sqrt{3})より、\mathbb{Q}(\sqrt{2},\sqrt{3})は正規拡大なので、ガロア拡大$\\
      $また、x^2-3は\mathbb{Q}(\sqrt{2})上の最小多項式でもあるので、拡大次数は[\mathbb{Q}(\sqrt{2},\sqrt{3});\mathbb{Q}] = [\mathbb{Q}(\sqrt{2},\sqrt{3});\mathbb{Q}(\sqrt{2})] \ [\mathbb{Q}(\sqrt{2});\mathbb{Q}] = 4となる。$

      \item $\forall \sigma \in Gal(\mathbb{Q}(\sqrt{2},\sqrt{3}) / \mathbb{Q})に対して、\sigma(\sqrt{2})は \sqrt{2} のK上の共役となるので、\sigma(\sqrt{2}) = \pm \sqrt{2}. \ 同様に、\sigma(\sqrt{3})は \sqrt{3} のK上の共役となるので、\sigma(\sqrt{3}) = \pm \sqrt{3}$\\
      $よって、
      id,
      \sigma \ (\sigma(\sqrt{2}) = -\sqrt{2},\sigma(\sqrt{3}) = \sigma{3}),
      \tau \ (\tau(\sqrt{2}) = \sqrt{2},\tau(\sqrt{3}) = - \sqrt{3}),
      \phi \ (\phi(\sqrt{2}) = -\sqrt{2},\phi(\sqrt{3}) = - \sqrt{3}),
      \  Gal(\mathbb{Q}(\sqrt{2},\sqrt{3}) / \mathbb{Q})の元であり、拡大次数が4であることからこれがGal(\mathbb{Q}(\sqrt{2},\sqrt{3}) / \mathbb{Q})の全ての元である。$
    \end{itemize}
    $以上から、Gal(\mathbb{Q}(\sqrt{2},\sqrt{3}) / \mathbb{Q}) = \{1,\sigma,\tau,\phi\}$


    \item $ガロア理論の基本定理より、中間体\mathbb{Q}(\sqrt{2}) / \mathbb{Q}に対応するガロア群はGal(\mathbb{Q}(\sqrt{2},\sqrt{3})/\mathbb{Q}(\sqrt{2}))である。$
    $[\mathbb{Q}(\sqrt{2}, \sqrt{3}) ;\mathbb{Q}(\sqrt{2}] = 2より\mid Gal(\sqrt{2}, \sqrt{3})/\mathbb{Q}(\sqrt{2}) \mid = 2.$ \\
    $よって、Gal(\mathbb{Q}(\sqrt{2}, \sqrt{3})) ;\mathbb{Q}(\sqrt{2}})は位数2の部分群である。$\\
    $Gal(\mathbb{Q}(\sqrt{2}, \sqrt{3})) / \mathbb{Q}(\sqrt{2}))は\mathbb{Q}(\sqrt{2})の元を不変にするので、Gal(\mathbb{Q}(\sqrt{2}, \sqrt{3})) /\mathbb{Q}(\sqrt{2}})) = \{id, \tau \}$

    \item $H=\{id,f\}とおく.ガロア理論の基本定理から、\mathbb{Q}(\sqrt{2}, \sqrt{3})^HがHに対応する中間体である。
    \mathbb{Q}(\sqrt{2}, \sqrt{3})^H = \{\alpha \in \mathbb{Q}(\sqrt{2}, \sqrt{3}) \mid \forall \sigma \in H, \ \sigma(\alpha) = \alpha \} = \mathbb{Q}(\sqrt{3}) $\\

    $*\{id,f\}は\mathbb{Q}(\sqrt{3})の元を固定するので、対応する中間体は\mathbb{Q}(\sqrt{3})$ \\

  \end{enumerate}

  \begin{description}
    \item[中間体との対応のまとめ] \mbox{}
    \begin{enumerate}
      \item 中間体が与えられたときにガロア群を求める方法の概要\\
      中間体を固定するようなK-自己同型群を求める
      \item ガロア群が与えられたときに中間体を求める方法の概要\\
      与えられたガロア群で固定されている中間体を求める
    \end{enumerate}
  \end{description}
\end{description}

\begin{tcolorbox}[colframe=black!50,colback=white,colbacktitle=black!50,coltitle=white,fonttitle=\bfseries\sffamily,title=問3]
$K=\mathbb{Q}(\sqrt{2}), \  L=\mathbb{Q}({}^4\sqrt{3})とする。全問と同様,Kは\mathbb{Q}のガロア拡大である。$
\begin{enumerate}
  \item $LはKのガロア拡大であることを示せ。$
  \item $Lは\mathbb{Q}のガロア拡大出ないことを示せ。$
\end{enumerate}
\end{tcolorbox}


\end{document}
